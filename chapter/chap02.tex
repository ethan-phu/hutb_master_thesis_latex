\chapter{研究内容1}
    \thispagestyle{mainstyle} % \chapter后面必须加这条命令启动页眉页脚设置
    \section{介绍}
    介绍。介绍。介绍。介绍。介绍。介绍。

    介绍。介绍。介绍。介绍。介绍。介绍。

    介绍。介绍。介绍。介绍。介绍。介绍。

    \section{相关工作}
    相关工作。相关工作。相关工作。相关工作。

    相关工作。相关工作。相关工作。相关工作。

    \section{方法}
    方法。方法。方法。方法。方法。方法。方法。

    方法。方法。方法。方法。方法。方法。方法。

    公式\ref{eq:eq1}。

    \begin{equation}
        a^2 + b^2 = c^2 \label{eq:eq1}
    \end{equation}

    图片\ref{fig:image1}。
    \begin{figure}
        \centering
        \includegraphics[width=0.8\textwidth]{images/image.jpeg}
        \caption{××图}
        \label{fig:image1}
    \end{figure}

    三线表格\ref{tab:table1}。
    \begin{table}
        \centering
        \caption{**表}
        \label{tab:table1}
        \begin{tabular}{c c}
            \toprule[1.5bp]
            列1 & 列2 \\
            \midrule[0.75bp]
            值1 & 值2 \\
            值1 & 值2 \\
            \bottomrule[1.5bp]
        \end{tabular}
    \end{table}

    如果表格的内容太多,可以把表格的文字缩小,不到迫不得已不要使用表格的续表形式。长表格的续表形式\ref{tab:long_table1}。如果长表格的多个表格之间插入了后文的内容,灵活地调整表格的长度,使得一个表格占一页。或者灵活地调整浮动体table的位置。或者调整后文(看tex文件)。
    \begin{table}[!t]
        \centering
        \caption{长表格}
        \label{tab:long_table1}
        \begin{tabular}{c c}
            \toprule[1.5bp]
            列1 & 列2 \\
            \midrule[0.75bp]
            值1 & 值2 \\
            值1 & 值2 \\
            值1 & 值2 \\
            值1 & 值2 \\
            值1 & 值2 \\
            值1 & 值2 \\
            值1 & 值2 \\
            值1 & 值2 \\
            值1 & 值2 \\
            值1 & 值2 \\
            值1 & 值2 \\
            值1 & 值2 \\
            值1 & 值2 \\
            值1 & 值2 \\
            值1 & 值2 \\
            值1 & 值2 \\
            值1 & 值2 \\
            值1 & 值2 \\
            值1 & 值2 \\
            值1 & 值2 \\
            值1 & 值2 \\
            值1 & 值2 \\
            值1 & 值2 \\
            值1 & 值2 \\
            值1 & 值2 \\
            值1 & 值2 \\
            值1 & 值2 \\
            值1 & 值2 \\
            值1 & 值2 \\
            \bottomrule[1.5bp]
        \end{tabular}
    \end{table}

    %\clearpage % 取消该行注释,同时注释下表后面的\clearpage,看看效果

    \begin{table}[!t]
        \centering
        \caption*{表\thetable \quad 长表格(续)}
        \begin{tabular}{c c}
            \toprule[1.5bp]
            列1 & 列2 \\
            \midrule[0.75bp]
            值1 & 值2 \\
            值1 & 值2 \\
            值1 & 值2 \\
            值1 & 值2 \\
            值1 & 值2 \\
            值1 & 值2 \\
            值1 & 值2 \\
            值1 & 值2 \\
            值1 & 值2 \\
            值1 & 值2 \\
            值1 & 值2 \\
            值1 & 值2 \\
            值1 & 值2 \\
            值1 & 值2 \\
            值1 & 值2 \\
            值1 & 值2 \\
            值1 & 值2 \\
            值1 & 值2 \\
            值1 & 值2 \\
            值1 & 值2 \\
            \bottomrule[1.5bp]
        \end{tabular}
    \end{table}

    \clearpage % 上面是长表格,之后的内容直接在新页面显示

    \section{实验}
    实验。实验。实验。实验。实验。

    实验。实验。实验。实验。实验。
    